%!TEX program = xelatex
% 完整编译: xelatex -> bibtex -> xelatex -> xelatex
\documentclass[lang=cn,11pt,a4paper]{elegantpaper}

\title{蓝眼睛岛问题}
\author{余周炜}


\date{\zhtoday}
\begin{document}
\maketitle

该问题是华人数学家陶哲轩最感兴趣的逻辑问题。

\section{问题描述}

(蓝眼睛岛问题)在一个岛上,住着一个部落。这个部落由1000个人组成,这些人有\textbf{多种}眼睛颜色。他们有一种宗教信仰,禁止他们了解自己眼睛的颜色,甚至禁止他们讨论眼睛颜色这个话题。因此,每个居民都能看到所有其他居民的眼睛颜色,但却不能发现自己的眼睛颜色(这里没有镜子或者诸如此类的东西)。如果有岛民知道了自己的眼睛颜色,那么他们的宗教就会强制他们第二天中午在村庄广场上自杀,让所有人都看到。所有的岛民都\textbf{非常逻辑和非常虔诚},他们\textbf{也都知道其他人也非常逻辑和非常虔诚}。在这里,“非常逻辑”的意思是,如果从岛民已知的信息观察中可以推理出任何结论,那么岛民必定会知道这个结论。

实际上,在这1000个岛民中,100个人的眼睛是\textcolor{blue}{蓝色},900个人的眼睛是\textcolor{brown}{棕色}。不过,岛民最初并不知道这些数据。因为,每个人只能看到除自己之外的999个人的眼睛颜色,看不到自己的眼睛颜色。

有一天,一个蓝眼睛的游客来到了岛上,并且获得了部落的完全信任。一天晚上,他向整个部落的人致辞,感谢他们的款待。

然而,由于不了解当地的风俗习惯,这位游客犯了一个错误,在他的讲话中提到了眼睛的颜色。他说了一句:“真是有趣呀,在这里能看到像我这样的蓝眼睛的人!”

请问,游客的失言对部落会不会产生影响?如果有,是什么影响?

\newpage

\paragraph{附加问题}
\begin{enumerate}
\item 如果游客在发言后的第二天就意识到了自己的错误,那么他有没有办法减少伤亡呢?
\item 如果游客意识到自己的错误不是在发言后的第二天,而是又过了几天之后,他有没有办法减少伤亡呢?
\end{enumerate}


\section{简单思考}

这里用到的是递归推理。

如果只有一个蓝眼人,那么他原本看不到蓝眼人,那么他听到游客的话就知道了游客说的这个蓝眼人就是自己,因此他在一天以后就会自杀。

如果有两个蓝眼人,那么他们原本只能看到一个蓝眼人,听到游客的话后,他们就会想对方是不是岛上唯一的蓝眼人呢,如果是的话,那么一天以后他也应该自杀了,但是过了一天没人自杀,因此两个人都明白了,对方没有自杀,是因为岛上还有一个蓝眼人,这个蓝眼人是谁呢?那么就是自己,因此在游客讲话的两天以后这两个人会一起自杀。

继续推理,如果有n个蓝眼人,那么这n个蓝眼人会在n天以后一起自杀。

我们也可以按时间推理。

过一天推理一次,那过了一天没人自杀,这说明蓝眼人的人数不是1;过了两天没人自杀,就说明蓝眼人的人数不是2;如此,过了n天没人自杀,那么蓝眼人的人数不是n-1。因此第n个蓝眼人就是自己,所以n个蓝眼人就在n天之后一起自杀。

\section{核心问题}

\paragraph{游客有没有带来新信息?}

答案是游客确实带来了新信息。

对于任何一个蓝眼人而言,他不妨把自己称为$A_1$,把其他99位蓝眼人称为$A_2,A_3,\cdots,A_{100}$,现在游客给$A_1$带来了新信息,如下所述:

\begin{quote}
$A_1$知道,$A_2$知道,$A_3$知道\ldots $A_{100}$知道岛上有蓝眼人。
\end{quote}

\paragraph{游客的话为什么能够带来新信息?}

因为他做的是一个公布,一个广播,即使他公布的内容是他早已知道的,但是公布这个动作本身就会造成一个重要的区别,就是现在所有人不但知道了公布的内容,而且知道所有人都知道,还知道所有人都知道所有人都知道\ldots 如此等等,以至于无穷。

\section{共识的分类}

\begin{itemize}
\item 弱共识(Mutual knowledge):如果一件事是所有人都知道的,那么这件事称为Mutual knowledge,也叫沉默共识。
\item 强共识(Common knowledge):如果一件事是所有人都知道,所有人都知道所有人都知道\ldots,\textbf{以至于无穷},那么称这件事为Common knowledge,也叫公开共识。
\end{itemize}

\section{该问题的十层思考}

\subsection{该问题本身的答案是什么?}

如果岛上有n个蓝眼人,游客说话之后n天,所有n个蓝眼人一起自杀。

\begin{proof}(数学归纳法)
\begin{enumerate}
\item $n=1$时,即如果岛上只有一个蓝眼人,那么他原本看不到蓝眼人,那么他听到游客的话就知道了游客说的这个蓝眼人就是自己,因此他在一天以后就会自杀。
\item 假设。对于$n=k$这个命题是成立的,如果岛上有k个蓝眼人,游客说话之后k天,所有k个蓝眼人一起自杀。
\item 现在我们考虑$n=k+1$的情况,这$k+1$个蓝眼人中的每一个都会做这样的推理,我看到了$k$个蓝眼人,假如他们就是全部的蓝眼人,那么过k天之后他们就会一起自杀,假如他们没有在k天之后一起自杀,那就说明蓝眼人不止k个,那唯一可能的其他蓝眼人是谁呢?就是他自己,于是他等待k天,然后看到没有人自杀,由此得出结论,自己是蓝眼人。每一个蓝眼人都做出了这样的推理,都知道了自己是蓝眼,因此$k+1$个蓝眼人一起自杀。
\end{enumerate}
\end{proof}

\subsection{棕眼人会不会自杀?}

不会,因为他们不知道岛上的眼睛颜色只有两种。

从推理过程讲,他们对眼睛做的推理只是在两种可能性中选择:蓝色或者不是蓝色,如果一个人确认自己眼睛不是蓝色,那么仍然可能是棕色,可能是红色,可能是黑色等等;因此他还是不知道自己眼睛的颜色,所以不会自杀。

\subsection{如果没有游客,岛民们会不会自杀?}

不会,因为没有推理的起点。 

\subsection{为什么我们要关心游客有没有带来新信息这个问题}

只有新的信息才能造成新的结果,这是一个常识,对于逻辑系统,这也是一个真理。

但是蓝眼睛岛问题似乎推翻了这个常识,因为一眼看不出来游客带来了什么新信息,我们需要搞清楚,错误的是这个常识呢,还是游客没有带来新信息这个印象。

\subsection{游客到底有没有带来新信息?}

确实有,新信息不是游客的话:岛上有蓝眼人,而是一个n阶的知识。

如果把这n个蓝眼人称为$A_1,A_2,\ldots,A_100$,那么新信息就是下面这个命题:

$K_n=A_1$知道,$A_2$知道,$A_3$知道,\ldots,$A_n$知道“岛上有蓝眼人”。

一个命题中出现了多少次知道,我们就把它叫做多少阶知识。

下面我们把“岛上有蓝眼人”这个命题记为$P$,

$A_n$知道$P$是一个一阶知识,记为

$K_1$=$A_n$知道$P$,

$A_{n-1}$知道$K_1$是一个二阶知识,记为

$K_2$=$A_{n-1}$知道$K_1$,如此等等

那么游客带来的新信息即为:

$K_n=A_1$知道$K_{n-1}$,这是一个$n$阶知识。

\subsection{为什么这个n阶的知识是一个新信息?}

在游客公布之前,通过互相观察岛民们最多只能获得$n-1$阶知识,不能知道$n$阶知识。因为每个人都不知道自己眼睛的颜色,他在推测别人怎么想的时候,只能考虑别人之间的相互观察,不能考虑别人观察自己的结果, 这就意味着每多一重知道,知识每升高一阶,那作为知识来源的人数就减少1,例如对于$K_1$,$A_n$知道$P$,$A_n$可以通过观察其他$n-1$个蓝眼人中的任何一个来推出$P$,那么作为知识的来源就是$n-1$,只需要扣除自己就行,对于$K_2$就是$A_{n-1}$知道$K-1$,那么$A_{n-1}$就需要扣除自己和$A_n$,那这个知识的来源就只有$n-2$个人,以此类推,对于$K_{n-1}$就是$A_2$知道$K_{n-2}$,$A_2$就需要扣除从自己到$A_n$的所有人,也就是$n-1$个人,那么知识的来源只剩下了一个人,就是$A_1$,对于$K_n$,n个人都被扣除了,都不能作为知识的来源,但是一个知识至少要有一个来源,因此$K_n$根本就不是个知识,由此可见,通过互相观察,是无法得到$K_n$的,最多只能得到$K_{n-1}$,那么游客发言之后,为什么就能得到$K_n$了,因为这时游客成为了一个公开的信息来源,每个岛民都可以从他那知道P,不再需要通过互相观察推来推去了,而且每个岛民都知道,其他人都从游客那里知道了P,因此岛民们不但是一下子知道了$K_n$,而且是一下子知道了$K_{n+1},K_{n+2}$,以至于任意高阶的知识,不过呢岛上只有n个蓝眼人,如果知识高于n阶,里面出现的人就必定有重复的,这就没有多少价值。

\subsection{新信息是如何发挥作用的?}

新信息使岛民们有可能通过一个判决性实验,来推出自己眼睛的颜色,这个判决性实验的做法就是等待和观察。

\subsection{在逻辑学上如何描述游客发言的效果?}

游客通过公布一个岛民们早已知道的信息,把它从弱共识提升到了强共识。这个操作产生的效果不是公布那个信息本身,而是一个n阶的知识。

\subsection{如何解答蓝眼睛岛问题的各种变体?}

如果游客发言之后已经过了m天,那么游客就需要指出m个人是蓝眼,或者带走m个人,才能阻止m个人全体自杀。

\subsection{哪些现象可以用这些道理来理解呢?}

如皇帝的新装,语言通讯模型,摩尔定律等。

\section*{参考资料}

Guan Video工作室的三个视频:
\begin{enumerate}
\item 做了这道数学题,可能会有减肥效果\ldots\ldots
\item 什么叫强共识?比如说:数学特别有趣
\item 科技袁人:如果说思考能力分为10层,你能达到第几层
\end{enumerate}

\end{document}
